\documentclass[12pt]{article}

\usepackage{sbc-template}
\usepackage{graphicx,url}
\usepackage{listings}
\usepackage{longtable}

\usepackage[brazil]{babel}   
%\usepackage[latin1]{inputenc}  
\usepackage[utf8]{inputenc}  
% UTF-8 encoding is recommended by ShareLaTex

     
\sloppy

\title{Codificador/decodificador convolucional usando o Algoritmo de Viterbi}

\author{Gabriel Batista Galli\inst{1}, Matheus Antonio Venancio Dall'Rosa\inst{1}, Vladimir Belinski\inst{1}}

\address{Ciência da Computação -- Universidade Federal da Fronteira Sul
  (UFFS)\\
  Caixa Postal 181 -- 89.802-112 -- Chapecó -- SC -- Brasil
  \email{\{g7.galli96, matheusdallrosa94, vlbelinski\}@gmail.com}
}

\begin{document} 

\maketitle
     
\begin{resumo} 
  O presente trabalho, apresentado ao curso de Ciência da Computação da Universidade Federal da Fronteira Sul - UFFS - Campus Chapecó - como requisito parcial para aprovação no Componente Curricular Inteligência Artificial, 2017.1, sob orientação do professor José Carlos Bins Filho, consiste em uma descrição da implementação de um codificador/decodificador convolucional usando o Algoritmo de Viterbi, bem como dos resultados obtidos.
\end{resumo}

\section{Descrição geral do algoritmo}
\section{Descrição dos problemas e soluções usadas}
\section{Exemplos de codificação/decodificação alcançados pelo programa}

% \bibliographystyle{sbc}
% \bibliography{sbc-template}

\end{document}